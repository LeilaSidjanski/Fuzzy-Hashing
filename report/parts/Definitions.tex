\section{Definitions}
\begin{description}
    \item[Equal Error Rate (EER)] \label{def:EER} Metric used to evaluate the performance of a system. It represents the point at which the system's \hyperref[def:FAR]{false acceptance rate (FAR)} equals its \hyperref[def:FRR]{false rejection rate (FRR)}. A lower EER indicates a more accurate and reliable system as it signifies a balanced trade-off between security (minimizing FAR) and usability (minimizing FRR)

    \item[False Acceptance Rate (FAR)] \label{def:FAR} This is the probability of incorrectly accepting an unauthorized user

    \item[False rejection Rate (FRR)] \label{def:FRR} This is the the probability of incorrectly rejecting an authorized user

    \item[Hash Function] \label{def:Hash_Function} A hash function is an algorithm that converts input data of any size to a smaller fixed-size string of characters, which typically acts as a data fingerprint. The output, known as a hash, is unique for different inputs in ideal cases, making hash functions crucial for cryptography, data integrity, and indexing in databases.

    \item[Fuzzy Extractors] \label{def:Fuzzy_Extractors} Fuzzy extractors are cryptographic tools designed to reliably and securely generate a consistent, reproducible cryptographic key from biometric data or other noisy inputs that are inherently inconsistent. They enable the extraction of a stable key from an input that may vary slightly over different measurements, ensuring that even with minor variations, the same key can be reliably regenerated. This process typically involves two main components: a generator that produces a stable key and some public data from an initial input, and a reproducer that can regenerate the original key from a similar but not identical input using the public data.
\end{description}
