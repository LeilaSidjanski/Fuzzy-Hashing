\section{Fuzzy Hashing}
\label{sec:Fuzzy Hashing}
Fuzzy hashing produces consistent cryptographic keys for similar but not identical inputs, enabling recognition of the same biometric trait across different instances despite slight variations. This approach ensures legitimate users are not incorrectly denied access due to minor discrepancies and protects user privacy by storing and using hashed values instead of raw biometric data, making it difficult to reverse-engineer the original data even if unauthorized access occurs. 
In this section, we will delve into the fuzzy hashing algorithm, its corresponding mathematical aspects, conduct experiments, and discuss the results obtained.
\subsection{Algorithm Implementation}

The fuzzy hashing algorithm, known as \textit{preHash}, operates on three inputs: a parameter \(m\), a key \(key\), and an image of finger veins \(X\). It produces a \(m\)-tuple containing the \(m\) smallest indices \(i_j\) such that \(1 <= i_1 < ... < i_m\), and the corresponding pixel \(PRNG_{key}(i_j)\) in \(X\) is identified as vein pixel.

In order to generate the tuple of indices, understanding the mechanics behind its creation is crucial. At the core of this system lies the pseudo-random number generator (PRNG), which operates by producing numbers uniformly and independently. This PRNG mechanism relies on the AES (Advanced Encryption Standard) block cipher functioning in counter (CTR) mode. AES processes data in fixed-size \(128\)-bit blocks. Leveraging AES' crpytographic properties, CTR mode combines a nonce (number only used once) with a counter, encrypting them with a secret key to generate a stream of seemingly random numbers. 
The predictability of this sequence is entirely dictated by the chosen key and nonce. In essence, employing the same key and nonce combination will consistently yield an identical sequence of numbers. 

To generate the nonce, we compute the SHA-256 hash of the key, resulting in a \(256\)-bit hash value. From this hash, we extract a \(128\)-bit value by slicing the first \(16\) bytes. Should a nonce already exist for a given instance, the system seamlessly preserves and reuses it. 

Upon receiving the parameters —key, nonce, and counter— CTR mode produces a \(128\)-bit pseudo-random number. However, to tailor the output to our requirements, we must mask it to \(17\) bits. This adjustment is essential because we aim to generate pseudo-random numbers within the range suitable for an image size, specifically \(96'500\) pixels, which necessitates \(17\) bits for representation. \textcolor{red}{(Faudrait-il écrire cela de manière plus générale? P.ex: produces a 128-bit prng which is then masked to an n bits...?)}

The algorithm, preHash, generates \(m\) smallest indices, denoted by \(i_j\), such that \(j\in{[1, m]} \) and \(1 <= i_1 < ... < i_m\), where each \(i_j\) correponds to an index such that \(X_{PRNG_{key}(i_j)} = 1\). It achieves this by tracking visited indices to maintain uniqueness and avoid infinite loops, and by rigorously verifying that the PRNG stays within the specified bounds (\(< 96'500\) \textcolor{red}{n??}).   

\textcolor{red}{mettre le pseudocode?}

\subsection{Mathematical Concepts}
\textcolor{red}{Not sure how to explain the math, its relevance et comment faire des liens pour passer d'une équations à une autre}

After processing within the pipeline and application of the \textit{preHash} algorithm to the finger vein data, the resulting output consists of indices. 

In the case where the parameters are m = 1, key is random, and k is uniformly distributed random index, we have: 

\begin{equation} \label{eq:preHash1}
    \begin{aligned}
        Pr[preHash_{key}^1(X) = preHash_{key}^1(Y)] &= \sum_{i > 0} Pr[preHash_{key}^1(X)\\
        &= preHash_{key}^1(Y)]\\
        &= \sum_{i > 0} Pr[X_k = Y_k = 0]^{i - 1} Pr[X_k = Y_k = 1]\\
        &= \frac{Pr[X_k = Y_k = 1]}{1 - Pr[X_k = Y_k = 0]}\\
        &= \frac{HW(X \land Y)}{HW(X) + HW(Y) - HW(X \land Y)}\\
        &= \frac{1}{\frac{1}{Score(X, Y)} - 1}
    \end{aligned}
\end{equation}

We notice that there is a direct link with the Miura matching score that we are interested in. The above computation can also be expressed as follows:

\begin{equation} \label{eq:preHash2}
    \begin{aligned}
        Pr[preHash_{key}^1(\bar{X}) = preHash_{key}^1(\bar{Y})] &= \frac{Pr[\bar{X}_k = \bar{Y}_k = 1]}{1 - Pr[\bar{X}_k = \bar{Y}_k = 0]}\\
        &= \frac{\frac{Pr[X_k = 1] + Pr[Y_k = 1]}{2} - \frac{1}{2}Pr[\bar{X}_l \neq \bar{Y}_k]}{\frac{Pr[X_k = 1] + Pr[Y_k = 1]}{2} + \frac{1}{2}Pr[\bar{X}_l \neq \bar{Y}_k]}\\
    \end{aligned}
\end{equation}


Inspired by equations \ref{eq:proba} and \ref{eq:delta}, we make the following approximations:

\begin{equation}
    E\left(\frac{\frac{Pr[X_k = 1] + Pr[Y_k = 1]}{2} - \frac{1}{2}Pr[\bar{X}_l \neq \bar{Y}_k]}{\frac{Pr[X_k = 1] + Pr[Y_k = 1]}{2} + \frac{1}{2}Pr[\bar{X}_l \neq \bar{Y}_k]}\right) \approx \frac{p - \frac{\delta}{2}}{p + \frac{\delta}{2}}
\end{equation}

Hence for (\(X\), \(Y\)) random,

\begin{equation}
    Pr[preHash_{key}^1(offset_X * X) = preHash_{key}^1(offset_Y * Y)] \leq \frac{p - \frac{\delta}{2}}{p + \frac{\delta}{2}}
\end{equation}

where equality is reached for the optimal offset translations. 

Depending on the distribution of (\(X\), \(Y\)), we denote

\begin{equation} \label{eq:mu}
    \mu = \frac{p - \frac{\delta}{2}}{p + \frac{\delta}{2}}
\end{equation}

We have the following figures:

\begin{table}[H]
    \centering
    \renewcommand{\arraystretch}{1.25}\begin{tabular}{|c|c|c|}
        \hline
        $\mu_{same}$ & $\mu_{diff}$ & $\mu_{indep}$\\
        \hline
        $24\%$ & $8.3\%$ & $1.8\%$\\
        \hline
    \end{tabular}
\caption{Comparison of Distributions: $\delta_{same}$, $\delta_{diff}$, and $\delta_{indep}$}
\end{table}

Finally, we have

\begin{equation}
    Pr[preHash_{key}^m(offset_X * X) = preHash_{key}^m(offset_Y * Y)] \leq \mu^m
\end{equation}

where equality is reached for the optimal offset translations.

%This includes determining the upper limits for the probabilities of similarity between different finger veins processed through the same fuzzy hashing parameters. 
\subsection{Part 3}

% Start this subsection by introducing the mathematical and theoretical concepts that underpin fuzzy hashing. Discuss the relevance of these concepts in the context of biometric data, focusing on how they enable the creation of reliable and secure hashing mechanisms for inherently noisy data.

%     Key Concepts to Cover:
%         Definition and significance of fuzzy hashing
%         Mathematical principles governing the construction of fuzzy hashes
%         Overview of the biometric setting, including the importance of parameters such as pixel dimensions, vein extraction, and the role of random permutations in hashing

% Subsection 2: Experimental Approach

% In the second subsection, outline the methodology of your experiments designed to test the theoretical underpinnings discussed earlier. Describe the setup, the specific objectives of each experiment, and how these experiments are structured to validate the theoretical models of fuzzy hashing.

%     Key Components to Include:
%         Description of the experimental setup and the data used
%         Explanation of how the experiments are designed to reflect the theoretical aspects of fuzzy hashing
%         Details on the implementation of preHash and postHash functions, and the criteria for their evaluation

% Subsection 3: Verifying Theoretical Predictions

% The final subsection is dedicated to comparing the outcomes of your experiments with the theoretical expectations. This involves analyzing the results, discussing any deviations or confirmations, and what these mean for the validity and reliability of fuzzy hashing in biometric data security.

%     Important Aspects to Discuss:
%         Analysis of experimental results against theoretical predictions
%         Discussion on the accuracy of the fuzzy hashing process, including the matching scores and error rates
%         Implications of the findings for biometric data security and future research directions

% Conclusion of Section 1

% Conclude with a summary of the insights gained from bridging theoretical concepts with empirical evidence. Highlight the importance of this integration for advancing the field of biometric security through fuzzy hashing. Reflect on the potential for future developments and applications stemming from your findings. --> Avons nous vraiment besoin de ça dans cette partie?